\documentclass{article}

\usepackage[margin=1in,left=1.5in,includefoot]{geometry}
%\usepackage{fancyhdr}
%\pagestyle{fancy}
%\fancyhead{}
%\fancyfoot{}
%\fancyfoot[R]{\thepage\}
%\renewcommand{\headrulewidth}{0pt}
%\renewcommand{\footrulewidth}{0pt}
\begin{document}

\begin{titlepage}
	\begin{center}
	\line(1,0){300}\\
	[0.25inc]
	\huge{\bfseries The Destribution and Nature of Agriculture in Uganda}\\
	[2mm]
	\line(1,0){200}\\
	[1.5cm]
	%\textsc{\LARGE OnlineTech96 Production}\\
	%[0.75cm]
	%\textsc{\large On the real Web System}\\
	%[9cm]
	\end{center}

	\begin{flushright}
	\textsc{\large MUWONGE TIMOTHY.\\
	\#14/U/23570/Eve \\
	\# 214023412\\
	May 14, 2017\\}
	\end{flushright}
\end{titlepage}

\cleardoublepage

\tableofcontents
\thispagestyle{empty}
\cleardoublepage
\pagenumbering{arabic}
\setcounter{page}{1}

\section{Introduction}\label{sec:intro}
Uganda is a land locked country located in the east African region which contains other countries namely Kenya, Tanzania, Rwanda, Burundi, and South Sudan. In east Africa, it is considered as third largest country. Security wise, Uganda has a peaceful political environment with no reported cases of terrorists in the past 10 years giving people a chance to live in one place time after time and year after year.

Being in the center of the tropical region, Uganda experiences a multitude of advantages of this region which include; reliable rainfall, plenty of sunshine, and fertile soils. Therefore, since all these favors the growth of agriculture, Uganda has turned out to be an agricultural country with a very large number of its population relaying on agriculture for both food and finance.

Agriculture commonly known as farming is the growth of crops and rearing of animals. I have chosen to dig deep and research about the extent to which agriculture is carried out in Uganda, its distribution and the nature of the agriculture (types of crops grown or animals reared).

\section{Keywords}\label{sec:keywords}
{\bfseries GDP} – Gross Domestic Product, {\bfseries PHC} - Population and Housing Census, {\bfseries  NGO} – Non-Government Organizations, {\bfseries  IFAD} - International Fund for Agricultural Development

\section{Background to the problem}\label{sec:background}
Being a landlocked equatorial region country, Uganda experiences reliable rainfall, reliable sun shine, possesses fertile soils, has a favorable climate with no tornados and snow, has a relatively flat relief, and also gifted with rich lakes and rivers most of which are none seasonal in nature. All these features have been a stepping stone for the growth of agriculture in Uganda since they all support agricultural activities in such a way that they provide the basic requirement necessary for farming.

Uganda is regarded as an agriculture-based economy and a food basket in the Eastern 
African region, given its ability to produce a variety of foods and in large quantities.    It 
comprises of the food and cash crops production,  livestock,  forestry  and  fishing  subsectors.    These  sub-sectors  contributed  62,  8,  17  and  13  percent  respectively  to 
agricultural  Gross  Domestic  Product  (GDP)  in  2011/12.    Agriculture  is  considered  an 
important  sector  that  contributed  23.7  percent  to  GDP  (at  current  prices)  in  2011/12. 
According to the  UCA  2008/9, there were approximately 3.95 million small and medium 
agricultural  households  with  a  population  of  19.3m  persons  (60\%  of  the  Uganda’s 
population) these produced the bulk (over 95 percent) of the food and cash crops.

The  agriculture  sector,  which  is  mainly  subsistence,  employs  the  largest  proportion  of 
Uganda’s work force. During the  Population and Housing Census (PHC)  2002,  about  73 
percent  (81  percent  female  and  67  percent  males)  of  the  work  force  was  employed  in 
agriculture,  making it the dominant  economic activity at that time.    The sector  remains  a
major  employer  to  date,  with  70  percent  and  66  percent  of  the  working  population 
engaged in agriculture during 2009/10 and 2010/11 respectively.  The sector is crucial for 
general  growth  of the  economy  (providing  inputs  into  the  industrial  sector)  and  poverty 
reduction especially among the rural poor for whom it provides employment.

After having gone through all the alleged current information about the agricultural sector in Uganda currently, I have chosen to investigate on the extent to which these allegations are true in otherward, to carry out a survey to determine the distribution of agriculture in Uganda   


\section{Problem Statement}\label{sec:problemstatement}


I want to develop a fully populated database having information of about 99\% of the agricultural farms in Uganda both in rural and urban areas. 
Today we are having many cases of inaccurate and inaccurate budgetary issues which leads to wastage of the tax payer’s revenue. If we ignore this problem there will be continued wastage of the government’s revenue as money will be wrongly directed.

The data to be collected about each farm is as follows;

Photo : The photo of the farm being investigated

GPS Coordinates : The GPS coordinates of the farm to cater for its location

District : District where the farm is located

Type of Farming : Either subsistance farming or commercial farming

Nature of Agriculture : either crops or animals

Kind of Crops/Animals : This represents the crops being planted in the farm ie banana , coffee or animals reared ie cows , hens

Size of Farm : the size is categorised as  Small scale , medium scale , large scale

Number of Workers : its a numeric value ( a decimal number )

We will use the farms database to help reduce on the cases of poor budgeting as valid and up to date information about the number, location and size of the farms will be available to the government when making the budgets


\section{Aim and Objectives}\label{sec:aimsandobjectives}
\subsection{Aim or general objective}
Aim:  To develop a fully populated database having all details about the farms in Uganda today and where they a located.

\subsection{Specific Objectives}

To move throughout the whole of Uganda identifying the Health facilities.

To find out the details needed from a farm in cases when a government is making a budget.

To design the database (physical and logical appearance).

To implement the design into a system.

To enter data into the database.

To test and validate the data.


\section{Research Scope}
\subsection{Geographical Scope}
	Since this research is country wide, this study will investigate the distribution and nature of farms in all districts starting from urban areas to rural areas for example from urban areas like Kampala and Wakiso to the deepest areas of Kalangala and Gulu districts to mention but a few 
\subsection{Theoretical Scope}
This system will actually be used by the system administrators excluding other organizations or personnels who just refer to the system. The excluded refer to the system by sending a request that is worked upon by the system administrators. The system administrators send back the output results in the various required formats such as excel document , pie charts , graphs and others
\section{Research Significance / Benefits}

The research findings and recommendations will be used by different organs or bodies for example UN to determine where to focus their funding in situations where need arises.

Agricultural Marketing organizations that are majorly NGOs will use my data to determine which product to put more effort in finding market for.

The research findings and recommendations will be used by the government in making the parliamentary budget 

Since it is a system with health details, it will help medical researches know the rate of diseases spread.

It will also be used to evaluate the impacts of the policies implemented by the government.


\section{References}



\end{document}